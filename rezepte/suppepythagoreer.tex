\begin{recipe}{Pythagoreer Suppe, 2 Personen}
	\ingredient{50\,g\,Polenta} in 
	\ingredient{100\,g Wasser} zubereiten und mit
	\ingredient{1 Eiersatz} vermengen, ziehen lassen und zu zwei Klößchen formen. 
	\modification{Die Klöße können auch angebraten werden}
	\ingredient{1 Gemüsebrühwürfel} in einem Kochtopf mit
	\ingredient{500\,ml\,Wasser} zum Kochen bringen.
	\item{Servieren} Suppe in zwei Schalen aufteilen und jeweils ein Kloß hinzufügen.
	\sidedish{Wasser oder Tee}
	\context Diese Suppe erinnert an die Pythagoreer, die Schüler von Pythagoras von Samos. Diese haben sich \enquote{frugal} ernährt und waren wohl gute Mathematiker, leider ist wenig von ihnen erhalten geblieben. Aber sie waren nicht perfekt, ihren Freund Archytas haben sie im Meer ertränkt, weil er die Irrationalität von $ \sqrt{2} $ erkannt hat. Irrationale Zahlen war für sie Teufelswerk. Die Suppe nutzt diese Geschicht um zur Reflexion einzuladen. Wie gehen wir, die doch Gelegenheit haben so viel zu lernen, wissen und verändern, mit Dingen um, die uns neuartig sind? Außerdem erinnern wir uns an die Märtyrer der Wissenschaft. Archytas ist das Klößchen in der Suppe. Die Suppe ist für Feierlichkeiten geeignet, zum Beispiel dem Julfest. Die Idee entstammt dem Buch Anathem.
\end{recipe}

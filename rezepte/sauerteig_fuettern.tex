\begin{recipe}{Sauerteig hegen \& pflegen}
	\item{Szenario} Der Sauerteig ist angesetzt und soll nun weiter geführt werden. Also immer ca. 80-90 g Anstellgut für den nächsten Backtag zu haben. 
	\ingredient{10 g Sauerteig} abnehmen und ihn mit 
	\ingredient{50 g Mehl} und 
	\ingredient{50 g Wasser}. Diese Mischung 8 bis 15 Stunden bei Raumtemperatur reifen lassen. 
	\item[Info] Dabei wölbt sich der Sauerteig nach oben. Reif ist er, wenn die Aufwölbung gerade dabei ist, ganz schwach einzufallen. Jetzt könnte er zu einem neuen Brotteig zugefügt werden. Zum aufbewahren, nicht so stark reifen lassen: Die Aufwölbung sollte also noch im Gange sein bis zum Backtag in den Kühlschrank stellen, ohne Probleme bis zu 14 Tage lang ohne Pflege aus.
	\item[Auffrischen] Ein Tag vor dem Backen von dem Sauerteigansatz aus dem Kühlschrank wieder 10 g ab nehmen, erneut mit je 50 g Mehl und Wasser mischen und reifen lassen. Dieses Mal aber vollständig, um ihn zum Backen zu verwenden. Von diesem aufgefrischten Anstellgut kann man nun die erforderliche Menge für meinen eigentlichen Sauerteig abnehmen.
	\item[Mehl] Für Weizen- und Roggensauerteig reichen als Behälter einfache Schraubgläser. Für den Weizensauerteigstarter verwendet man Weizenmehl 1050, für den Roggensauerteigstarter Roggenmehl 1370 (oder 1150).
	\item[Temperatur] Zum Führen eines Sauerteiges sollte zwischen 23 bis 32\textcelsius liegen. Um das Optimale aus dem Sauerteig herauszuholen, empfiehlt sich eine Dreistufenführung. Diese ist jedoch aufwändig und passt nicht immer in den Zeitplan. Ein Kompromiss wäre zum Beispiel, den angesetzten Sauerteig fallend von ca. 35\textcelsius bis auf Zimmertemperatur einfach 12-18 Stunden stehen zu lassen. Man kann z.B. seinen Backofen auf 50\textcelsius C aufwärmen und abschalten, den Sauerteig hineinstellen, für 1-2 Stunden die Lampe anschalten und ihn dann sich selbst überlassen bis er reif ist.
	\item[Tipp] Für kleine Mengen Sauerteig, etwa zum Auffrischen, verwende ich eine Styroporbox von ca. 25 x 25 x 25 cm (gab es mal zu einer gekühlten Bestellung dazu), in der 2-3 Warm-Kalt-Kompressen (ca. 10×20 cm groß, mit blauem Gel gefüllt) für Wärme sorgen. Die Kompressen übergieße ich zuvor mit kochendem Wasser und lasse sie 2-3 Minuten darin aufheizen. Anschließend trockne ich sie ab und lege sie mit etwas Abstand zu den Sauerteigbehältern in die Box. Über 6-8 Stunden lässt sich so eine Temperaturkurve von ca. 35°C abfallend auf 25°C erreichen, ideal für Sauerteig.
	\author{Plötz, \href{http://www.ploetzblog.de/tipps-und-tricks/sauerteig/}{http://www.ploetzblog.de/tipps-und-tricks/sauerteig/}}
\end{recipe}
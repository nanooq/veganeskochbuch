\begin{recipe}{Onigiri (gefüllte Reisbällchen)}
	\ingredient{500\,g\,Basmati Reis}  in 
	\ingredient{2\,EL\,Gemüsebrühe} unter ständigem Rühren zum Kochen bringen, bis die Brühe verkocht und der Reisklebrig ist. Anschließend
	\ingredient{10\,EL\,Parmesan} unterrühren und erkalten lassen. Nun wird die Füllung vorbereitet:
	\ingredient{300\,g\,Kartoffeln} in Salzwasser weich kochen, abgießen, zerstampfen und erkalten lassen. 
	\ingredient{2\,Zwiebeln} in einer großen Pfanne mit
	\ingredient{Öl} weichdünsten. Dazu kommen zum mitdünsten 
	\ingredient{3\,grüne\,Chili} in feine Ringe geschnitten,
	\ingredient{1\,Knoblauchzehen}, fein gehackt und 
	\ingredient{5\,g Ingwer}, zerrieben oder sehr, sehr klein geschnitten. Danach kommen die Gewürze:
	\ingredient{3\,TL\,Garam\,Masala},
	\ingredient{1\,TL\,gemahlener\,Koriander},
	\ingredient{1\,TL\,gemahlener\,Kreuzkümmel},
	\ingredient{$\frac{1}{3}$\,TL\,gemahlener\,Kurkuma},
	\ingredient{$\frac{1}{3}$\,TL\,gemahlener\,Cayenne-Pfeffer} einrühren und darauf achten, dass genügend Öl in der Pfanne ist. Danach alles mit Wasser ablöschen, die Kartoffeln und 
	\ingredient{4\,EL\,frische oder tiefgekühlte Erbsen} sowie
	\ingredient{1\,$\frac{1}{3}$\,Salz} zugeben. Alles gut mischen, bei kleiner Hitze ziehen lassen und abkühlen lassen. Die Füllung sollte ein eher trockener Brei sein. Eine kleine Reiskugel formen, mit dem Daumen eindrücken, diese Delle mit der Füllung füllen und mit weiterem Reis verschließen. Die entstehende Kugel sollte so groß wie ein Tennisball sein.
\end{recipe}
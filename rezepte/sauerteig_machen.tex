\begin{recipe}{Sauerteig anzüchten}
	\ingredient{ca. 100 g Mehl} Roggen-, Weizen- oder Dinkelmehl vom Typ 1050 oder 1150 mit 
	\ingredient{lauwarmes Wasser} verühren bis ein flüssiger Brei entsteht. Abdecken und warm stehen und gehen lassen. Alle 12 Stunden den Brei ordentlich durchschlagen, bis er Blasen wirft. Dann wieder abgedeckt warm stehen und gehen lassen.
	Jeden Tag 100 g Mehl und soviel lauwarmes Wasser beigeben, bis es eine waffelteigartige Masse wird.
	\item{Wiederholen} Solange bis der Getreidebrei heftig gärt, blubbert und sauer wird. Das riecht man, danach stabilisiert sich die Säure. Danach setzt die Verhefung ein, das kann man nicht sehen, aber riechen. Danach entsteht ein angenehmer Duft von frischem Quark, Zitrusfrüchten, Balsamico-Essig oder einem frisch aufgeschnittenem Apfel.
	\item[Achtung] Neben der gewollten Spontansäuerung gibt es auch die ungewollte durch fremde Bakterien und Pilze. Letzteres erkennt man an einer deutlichen roten, schwarzen, blauen oder grünen Verfärbung. Wenn der Teig wirklich ekelhaft oder sehr streng nach Essig stinkt oder wenn sich Schimmelhaare bilden. Dann muss der Ansatz leider entsorgt werden und das benutzte Geschirr gründlich mit heißem Wasser zu spülen, ordentlich trocknen und es mit anderem Mehl neu zu probieren. Oder sich eine stabile Sauerteigkultur zu besorgen.
\end{recipe}